%!TEX root = ../dokumentation.tex

\chapter{Beispiel Code-schnipsel einbinden}

%title wird unter dem Bsp. abgedruckt
%caption wird im Verzeichnis abgedruckt
%label wird zum referenzieren benutzt, muss einzigartig sein.

\begin{lstlisting}[caption=Code-Beispiel, label=Bsp.1]
public class HelloWorld {
	public static void main (String[] args) {
		// Ausgabe Hello World!
		System.out.println("Hello World!");
	}
}
\end{lstlisting}

%language ändert die Sprache. (Wenn nur eine Sprache verwendet wird, kann diese Sprache in einstellungen.tex geändert werden. Standardmäßig Java.)
\begin{lstlisting}[caption=Python-Code, label=Python-Code, title=Titel des Python-Codes,language=Python]
def quicksort(liste):
if len(liste) <= 1:
	return liste
pivotelement = liste.pop()
links = [element for element in liste if element < pivotelement]
rechts = [element for element in liste if element >= pivotelement]
return quicksort(links) + [pivotelement] + quicksort(rechts)
# Quelle: http://de.wikipedia.org/wiki/Python_(Programmiersprache)
\end{lstlisting}

\begin{lstlisting}[caption=Example program invoking \textit{system()}, label=system-example, title=Example program invoking \textit{system()},. language=C++]
#include "stdlib.h"

int main(){
system("echo $(hostname)");
}
\end{lstlisting}


\section{Verweis auf Code}
Verweis auf den Code \autoref{Bsp.1}.\\
und der Python-Code \autoref{Python-Code}.

Zweite Erwähnung einer Abkürzung \ac{AGPL} (Erlärung wird nicht mehr angezeigt)