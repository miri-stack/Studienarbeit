\chapter{Introduction}
Evolution is said to be a game. On the other hand, evolution takes place on a molecular level and quantum mechanics explains the behaviour of such small particles. Assuming both perspectives on evolution are valid, a connection of game theory and quantum physics exist-

---

Decision-making is formalised in game theory, a field of applied mathematics. Games consist of players, strategies and pay-off functions. 

..

Quantum mechanics can extend classical game theory. This extension has various forms. For instance, quantum operations can be applied on quantum bits as strategic moves.

(Those quantum bits could be information carriers.)

(Further, entangled particles allow for new ways of cooperation by serving as means of communication.)

Like classical game theory, quantum games can be applied to different areas of interest. 

Secure communication theory can be regarded as a game against the eavesdropper. With the development of quantum communication and quantum information carriers, this game has to be considered with means of quantum mechanics.

Non-local games are used in physics to answer the question whether the universe is infinite dimensional or not.

Research is made to prove lower bounds in algorithms with game theory. 

Es stellt sich die Fragen, inwiefern Quantenspiele neue Moeglichkeiten im Gegensatz zur klassischen Spieltheorie eroeffnen


---

Alice and Bob toss a coin. Both bet on one side of the coin and win some money if they were right. What if the coin was a quantum bit?