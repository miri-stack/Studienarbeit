\chapter{Theoretical Basis}
\section{Linear Algebra}
\section{Quantum Mechanics}
\begin{enumerate}
\item Notation from Nielsen und Chuang
\item Ket vector
\item bra vector
\item inner product
\item outer product
\item Tensor product
\item Hermitian conjugate or adjoint of the A matrix
\item Inner product between $\ket{\phi}$ and $A\ket{\psi}$. Equivalently, inner product between $A^\dagger\ket{\phi}$ and $\ket{\psi}$.
\end{enumerate}
\textcolor{red}{Todo: Intro Quantum}


Quantum mechanics uses the \textit{braket} notation. Instead of writing $\vec{\rho}$ to represent a state vector, it is denoted by a \textit{ket}, i.e. $\ket{\rho}$. \textit{Bra $\rho$} refers to the conjugate transpose of the vector, thus $\bra{\rho}=(\ket{\rho})^\dagger$. 

The state of a quantum system is described by a \textit{state vector} in its \textit{state space}. A state space is a complex Hilbert space. In case of a qubit, the state space is two-dimensional. An orthonormal basis for the state space of a qubit are the vectors $\ket{0}$ and $\ket{1}$. $\ket{0}$ is not to be confused with the \textit{zero vector} for which the braket notation is not used. 

\begin{equation}
    \ket{0}=\begin{pmatrix}1\\0\end{pmatrix}, \ket{1}=\begin{pmatrix}0\\1\end{pmatrix}
\end{equation}

Each state vector of a qubit can be represented as a linear combination of the basis vectors: \begin{equation}\ket{\rho}=\alpha\ket{0}+\beta\ket{1}=\begin{pmatrix}\alpha\\\beta\end{pmatrix}
\end{equation}

$\alpha$ and $\beta$ are called \textit{probability amplitudes}. State vectors are unit vectors and thus normalised. This yields to the condition $\braket{\rho\vert\rho}=(\alpha\bra{0}+\beta\bra{1})(\alpha\ket{0}+\beta\ket{1})=1$. With the inner product of a vector with itself being $1$, and the inner product of two orthogonal vectors being $0$, the equation can be further simplified. Thus, the amplitudes are limited by
$|\alpha|^2+|\beta|^2=1$.
If both amplitudes are unequal to zero, the quantum system is said to be in a  \textit{superposition} of states. 

Yet, one cannot measure such superposition. When measured, a quantum system collapses to the measured state. Considering the basis $\{\ket{0},\ket{1}\}$, either one of the two states is measured. It is also possible to perform measurements with respect to a different basis. 

\textbf{Born's Rule}

The quantum system is measured with respect to an orthonormal basis spanning its Hilbert space. Those are not necessarily $\ket{0}$ and $\ket{1}$ as it can be seen in chapter \ref{Eisert}. Consider the basis $\{\ket{\phi_1},...,\ket{\phi_n}\}$. The probability $P(\phi_i)$ of measuring a state $\ket{\phi_1}$ where $1\leq i\leq n$

For the introduced standard basis, the probabilities of measuring $\ket{0}$ and $\ket{1}$ are $|\alpha|^2$ and $|\beta|^2$ respectively. 



\subsection{Observables}

\subsection{Operations}

\subsection{Multiple Quantum Bits}
\begin{enumerate}
\item Basis states:
    how do they look like, tensor product for one
\end{enumerate}




\section{Game Theory}
Equivalence

Nash Equilibrium

Pareto Optimal

game categories